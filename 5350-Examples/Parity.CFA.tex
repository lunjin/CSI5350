\documentclass{beamer}
\usepackage{tikz}
\usetikzlibrary{trees,shapes.geometric, positioning, arrows}

\begin{document}

\tikzstyle{state} = [circle, inner sep=2pt, radius =20pt, text centered, draw=black,  fill=blue!50]
\tikzstyle{point} = [circle, inner sep=0pt, minimum size =1pt,fill]
\tikzstyle{myarrow} = [->, >=latex]
\begin{tikzpicture}[node distance = 2cm, >=latex]

\matrix[row sep =5em,column sep=5em]{
 &\node(vert1)[point]{}; & &\\
 &\node(vert6)[state]{A}; & &\node(vert2)[state]{E};\\
 &\node(vert5)[state]{B}; & &\\
\node(vert4)[state]{C}; & &\node(vert3)[state]{D}; &\\
};

\draw [myarrow]  (vert5)  -- node[sloped,above]{$n\%2=0$} (vert4);
\draw [myarrow]  (vert5)  -- node[sloped,above]{$n\%2 \neq 0$} (vert3);
\draw [myarrow]  (vert6)  -- node[sloped,above]{$n\neq 1$}  (vert5);
\draw [myarrow]  (vert6)  -- node[sloped,above]{$n=1$}  (vert2);
\draw [myarrow]  (vert4) edge[bend left]  node[sloped, above]{n:=n/2} (vert6);
\draw [myarrow]  (vert3) edge[bend right]  node[sloped, above]{n:=3*n+1} (vert6);
\draw [myarrow]  (vert1)  --  node[sloped, above]{} (vert6);

\end{tikzpicture}
\end{document}

